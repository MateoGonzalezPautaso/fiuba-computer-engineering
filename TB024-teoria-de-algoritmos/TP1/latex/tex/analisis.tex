\section{Análisis del problema}

Para comenzar a plantear la solución al problema, tomamos ideas de los ejercicios vistos en clase relacionados con Algoritmos Greedy, particularmente el ejercicio de la mochila y el de Scheduling II.\\

\subsection{Primer enfoque}

La primera idea consistió en considerar como regla Greedy el elegir en cada estado la batalla con mayor peso que aún no se había luchado. Esta propuesta surge de un análisis de la fórmula que calcula la suma ponderada a minimizar:

\begin{equation*}
    \sum_{i=1}^{n} b_i F_i
\end{equation*}

Dado que el valor de $F_i$ (el acumulado de felicidad) aumenta a medida que se acumulan batallas, parecía razonable que las batallas con mayor peso $b_i$ se multipliquen con los valores de felicidad más chicos. De esta forma, se interpretó que convenía realizar primero aquellas batallas más importantes.
Sin embargo, luego de un análisis, llegamos a la conclusión de que esta solución propuesta no es óptima, demostrado por este contraejemplo:

\begin{quote}
    Consideremos dos batallas:
    \[
    A: \; t_A = 3,\; b_A = 3
    \qquad
    B: \; t_B = 1,\; b_B = 2
    \]
    
    Aplicando esta solución, se prioriza elegir primero la batalla $A$ debido a ser la de mayor peso, lo que nos deja con el siguiente resultado en la suma ponderada:
    
    \begin{align*}
        Caso_{A \to B} &= b_A \cdot t_A + b_B \cdot (t_A + t_B) \\
                    &= 3 \cdot 3 + 2 \cdot (3+1) \\
                    &= 9 + 8 = 17 \\
    \end{align*}
    
    Siendo que la solución óptima es la siguiente:
    
    \begin{align*}
        Caso_{B \to A} &= b_B \cdot t_B + b_A \cdot (t_B + t_A) \\
                    &= 2 \cdot 1 + 3 \cdot (1+3) \\
                    &= 2 + 12 = 14
    \end{align*}
    
    Como $14 < 17$, se observa que priorizar sólo por el peso de la batalla no resuelve el problema de manera óptima.
\end{quote}


\newpage
\subsection{Segundo enfoque}

Luego de comprobar que la primera regla planteada no garantizaba optimalidad, decidimos desarrollar la solución a partir de la propia fórmula que calcula la suma ponderada. Para ello, consideramos el caso más simple con dos batallas:

\begin{quote}
    \[
    A: \; t_A,\; b_A
    \qquad
    B: \; t_B,\; b_B
    \]
    
    Suponemos que para minimizar el resultado final, la primera batalla a eligir debe ser $A$. ¿Qué debe pasar para que este sea el caso?
    
    Desarrollamos los dos casos de la sumatoria, en uno primero se elige $A$ y en el otro primero se elige $B$. Desde la suposición inicial se puede formular la próxima inecuación

    \begin{align*}
        b_A \cdot t_A + b_B \cdot (t_A + t_B) &< b_B \cdot t_B + b_A \cdot (t_A + t_B)
    \end{align*}

    Se aplica la distributiva en ambos lados:
    
    \begin{align*}
        b_A \cdot t_A + b_B \cdot t_A + b_B \cdot t_B &< b_B \cdot t_B + b_A \cdot t_A + b_A \cdot t_B 
    \end{align*}
    
    Cancelamos los términos redundantes:

    \begin{align*}
        b_B \cdot t_A &< b_A \cdot t_B 
    \end{align*}

    Lo que resulta en que:

    \begin{align*}
        \frac{t_A}{b_A} &< \frac{t_B}{b_B}
    \end{align*}

    En conclusión, para que haber elegido la batalla $A$ como la primera resulte en una sumatoria menor que el caso donde se elige primero la batalla $B$, el cociente $\frac{t_A}{b_A}$ debe haber sido menor que el cociente $\frac{t_B}{b_B}$
\end{quote}