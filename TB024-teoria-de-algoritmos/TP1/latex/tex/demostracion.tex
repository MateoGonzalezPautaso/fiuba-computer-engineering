\section{Demostración de optimalidad del algoritmo}

A continuación, se demostrará que seleccionar el orden de las batallas por el cociente $t_i/b_i$ de forma creciente produce una solución óptima para minimizar la sumatoria ponderada:
    \begin{equation*}
        \sum_{i=1}^{n} b_i F_i
    \end{equation*}

\subsection{Definición de inversión}

\paragraph{Definición.}
Decimos que una solución tiene una \emph{inversión} si existen dos posiciones contiguas $i$ y $j$, donde $i < j$ y, a su vez,
\[
\frac{t_i}{b_i} > \frac{t_j}{b_j}.
\]


\subsection{Planteo: existe un orden de batallas sin inversiones que es óptimo}

\paragraph{Demostración.}
Supongamos que hay un orden de batallas óptimo que sí tiene inversiones. Como las tiene, debería poder intercambiar siempre de a pares las batallas inversibles y contiguas. Luego de una cantidad de inversiones, se debería llegar a un orden sin inversiones que acumule el mismo coeficiente de impacto.\\
    A continuación se demuestra matemáticamente que hacer estas inversiones no puede empeorar la solución óptima. Es decir, realizar una inversión entre dos batallas no aumenta la sumatoria ponderada.

\begin{figure}[H]
    \centering
    \includegraphics[width=0.8\textwidth]{img/inversiones.png}
\end{figure}

Analizando cómo afecta la inversión en el coeficiente de impacto $C_i$:

\begin{quote}
    Sumatoria antes de la inversión ($K$ representa la sumatoria de las batallas luchadas hasta el momento y $F$ la felicidad acumulada hasta la primera batalla presente en la inversión): 
    \[ C_i= K + b_i \cdot (F + t_i) + b_j \cdot (F + t_i + t_j) \]
    
    
    Luego de hacer la inversión la sumatoria se representa con:

    \[ C_i'= K + b_j \cdot (F + t_j) + b_i \cdot (F + t_j + t_i) \]
    
    Se desarrollan ambas ecuaciones:
    \begin{align*}
        C_i = K + b_i \cdot F + b_i \cdot t_i + b_j \cdot F + b_j \cdot t_i + b_j \cdot t_j\\[4pt]
        C_i' = K + b_j \cdot F + b_j \cdot t_j + b_i \cdot F + b_i \cdot t_j + b_i \cdot t_i
    \end{align*}\
    
    Observando los términos de cada una, se visualiza que hay un único termino diferente entre ellas:

    \begin{align*}
        Término \ C_i \xrightarrow{} b_j \cdot t_i\\[4pt]
        Término \ C_i' \xrightarrow{} b_i \cdot t_j
    \end{align*}\

    Recordando que $\frac{t_i}{b_i} > \frac{t_j}{b_j}$ y que por ende ${t_i} \cdot {b_j} > {t_j} \cdot {b_i}$, se concluye que haber hecho una inversión no pudo haber empeorado la solución, es decir, provocar un aumento en el resultado final de la sumatoria.
    Esto demuestra que existe un orden de batallas óptimo sin inversiones.
\end{quote}

\subsection{Planteo: todos los órdenes de batallas sin inversiones son igual de óptimos}


\paragraph{Demostración.}
Las diferencias entre todos los órdenes de batallas que son óptimos solo pueden darse entre batallas contiguas que satisfacen \(\dfrac{t_i}{b_i}=\dfrac{t_j}{b_j}\)

Por este motivo, basta con probar que intercambiar el orden de esas batallas no altera el coeficiente de impacto final.

En un ejemplo con un intercambio entre dos batallas $i$, $j$, donde \(\dfrac{t_i}{b_i}=\dfrac{t_j}{b_j}\):

Sea $T$ el tiempo acumulado antes de estas dos tareas; las contribuciones a la sumatoria en cada orden toma la forma:
\[
\begin{aligned}
C_{i\to j} &= b_i(T+t_i) + b_j(T+t_i+t_j),\\
C_{j\to i} &= b_j(T+t_j) + b_p(T+t_j+t_i).
\end{aligned}
\]

Se aplica la distributiva para analizar cada término en cada una de las soluciones: 

\[
\begin{aligned}
C_{i\to j} &= b_i T + b_i t_i + b_j T + b_j t_i + b_j t_j, \\[6pt]
C_{j\to i} &= b_j T + b_j t_j + b_i T + b_i t_j + b_i t_i.
\end{aligned}
\]

Si $\dfrac{t_i}{b_i}=\dfrac{t_j}{b_j}$ entonces $b_j t_i = b_i t_j$, por lo que ambas soluciones son igual de óptimas.

Finalmente, cualquier permutación entre batallas contiguas y con el mismo cociente \(\dfrac{t}{b}\) no altera el resultado final. Entonces, todas las soluciones sin inversiones son igual de óptimas.


\subsection{Conclusión de la demostración}

Como hay un orden de batallas sin inversiones que es óptimo y, además, todos los órdenes de batallas sin inversiones generan el mismo coeficiente de impacto (demostrado en \textbf{2.1} y \textbf{2.2}). Concluimos que este algoritmo es óptimo debido a que logra encontrar un orden de batallas sin inversiones.

\newpage