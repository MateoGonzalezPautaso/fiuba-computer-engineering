\section{Demostración que el Problema de la Tribu del Agua se encuentra en NP.}

Primero, se plantea el problema de decisión correspondiente: dado un conjunto de maestros y un número $k$, ¿es posible separarlos en $k$ subgrupos tales que la suma de los cuadrados de las fuerzas totales de cada grupo sea, como máximo, un valor $B$?

Para demostrar que este problema está en NP, construimos un validador para verificar si la solución es correcta. Este validador tiene que funcionar en tiempo polinomial. \\

    \textbf{Parámetros del verificador}
    \begin{itemize}
        \item \textbf{maestros\_agua:} Representa el set de datos representando a los diferentes maestros, siendo un diccionario con el formato {nombre: poder}
        \item \textbf{k:} número entero que representa la cantidad de subgrupos en los que deben ser separados los maestros
        \item \textbf{B:} Es el valor máximo que puede alcanzar la adición de los cuadrados de las sumas de las fuerzas de los grupos.
        \item \textbf{subgrupos:} Es la solución a validar. Consiste en una lista de listas, cada subgrupo es una lista de nombres de maestros
    \end{itemize}
Dicho algoritmo se puede observar en la siguiente sección de código:

\begin{lstlisting}[language=Python]
def grupos_parejos(maestros_agua, k, B, subgrupos):
    maestros_dict = {nombre: poder for nombre, poder in maestros_agua}
    maestros_agua = maestros_dict
    
    if len(subgrupos) != k:
        return False    # La cantidad de subgrupos no coincide con k
    
    asignados = set()
    B_conseguido = 0

    for subgrupo in subgrupos:
        poder_subgrupo = 0
        
        for nombre_maestro in subgrupo:

            if nombre_maestro in asignados or nombre_maestro not in maestros_agua:
                return False    # El maestro solo puede estar asignado a un subgrupo
            
            asignados.add(nombre_maestro)
            poder_subgrupo += maestros_agua[nombre_maestro]

        B_conseguido += poder_subgrupo ** 2

    if len(asignados) != len(maestros_agua):
        return False    # Cada maestro debe estar asignado a un grupo

    return B_conseguido <= B
\end{lstlisting}


\textbf{Complejidad del Validador:}
    \begin{enumerate}
        \item Lo primero que se hace es convertir la lista de tuplas a un diccionario y luego verificar si la longitud de la solución recibida, es decir, la cantidad de subgrupos en los que se separaron los maestros, es igual al valor k pasado por parámetro: O(n) + O(1)
        \item A continuación, se procederá a revisar los subgrupos y, dentro de estos, se examinarán los maestros presentes, verificando la validez de las asignaciones y calculando el valor B de la solución obtenida.
        Como cada maestro debe estar asignado a un solo grupo, la cantidad de iteraciones que se van a realizar son O(n + k), siendo n la cantidad de maestros.
        \item Las verificaciones siguientes en el codigo no aportan complejidad adicional, ya que son O(1)
    \end{enumerate}

Por lo tanto, el comportamiento del validador de la solución para el problema de Maestros del Agua, es polinomial, concluyendo que el problema esta en NP.



