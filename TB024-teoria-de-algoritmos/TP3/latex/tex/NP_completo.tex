\section{Demostración que el Problema de la Tribu del Agua es NP-Completo.}

Para demostrar que el problema enunciado es, en efecto, NP-Completo, debemos reducir un problema NP-Completo a este. Para esta reducción elegimos reducir Partition-Problem a Tribu Del agua, lo que implica resolver Partition-Problem con un algoritmo que resuelva Tribu del Agua

\subsection{Demostración que Partition-Problem es NP-Completo}

Para poder demostrar que Partition-Problem es un problema NP-Completo, vamos a reducir un problema que ya sabemos que esta dentro de esta categoría al mismo. El problema elegido para esto es Subset-Sum

Primero se plantea el problema de decisión de Partition-Problem: Dado un conjunto X de n elementos, ¿Se puede dividir X en dos subconjuntos de igual suma?

\subsubsection{Pertenencia a NP}

A continuación se muestra un validador para la solución que funciona en tiempo polinomial, demostrando que este problema pertenece a NP.

\begin{lstlisting}[language=Python]
def partition_problem_valido(X, asignaciones):
    '''
    X: lista de elementos
    asignaciones: lista de 0s y 1s del mismo largo que X, donde 1 indica que el elemento va al subconjunto S1 y 0 que va a S2.
    '''
    
    if len(X) != len(asignaciones):
        return False
    
    suma_s1 = 0
    suma_s2 = 0

    for i in range(asignaciones):
        asig = asignaciones[i]
        if asig == 1:
            suma_s1 += X[i]
        elif asig == 0:
            suma_s2 += X[i]
        else:
            return False

    return suma_s1 == suma_s2
\end{lstlisting}

\subsubsection{Reducción de Subset-Sum a Partition-Problem}

Para lograr esta reducción, se debe poder resolver el problema de Subset-Sum usando un algoritmo que pueda resolver Partition-Problem, mediante una cantidad de pasos polinomial en el medio.
Para lograr esto, vamos a transformar el problema:

Los parámetros que va a recibir Subset-Sum son los siguientes:
\begin{itemize}
    \item X: como un conjunto de elementos $\{x_1,x_2, ..., x_n\}$
    \item t: un número objetivo
\end{itemize}

Estos parámetros se transformarán en los siguientes para enviárselos al algoritmo que resuelve Partition-Problem:
\begin{itemize}
    \item Primero, defino $ \sigma =  \sum_{xi \in X} x_i$
    \item $X' = X \cup \{2\sigma - t, \sigma+t\}$ = $\{x_1,x_2, ..., x_n, 2\sigma - t, \sigma+t\}$
    \item La suma de todos los elementos de $X'$ equivale a $4\sigma$
    \item Siguiente, resuelvo Partition-Problem, pasándole como parámetro a X'
\end{itemize}

Ahora se demuestra la ida y la vuelta de la reducción

\begin{enumerate}
    \item \underline{\textbf{Ida: Cuando hay subset-sum hay partition}}
    \item[] Como hay subset-sum, existe el Subconjunto $S \subseteq X$ cuya suma de elementos es el valor objetivo $t$.\\
            Por lo tanto, la suma de los elementos fuera de $S$ es $\sigma - t$. \\
            Ahora considero el subconjunto de $X'$ llamado $A = S \cup \{2\sigma - t\}$. La suma de sus elementos es $t + (2\sigma - t) = 2 \sigma$.\\
            El complemento de $A$ en $X'$ es $B$ = $(X' - A)$, que es lo mismo que: $B = (X - S) \cup \{\sigma + t\}$. La suma del subconjunto $B$ es lo que sobra de $X$ y el otro término agregado: $(\sigma - t) + (\sigma + t) = 2\sigma$. \\
            Como $A$ y su complemento $B$ suman lo mismo, el conjunto $X'$ es particionable. Por lo tanto, si existe solución al Subset-Sum, existe solución a Partition-Problem
    \item \underline{\textbf{Vuelta: Cuando hay partition hay subset-sum}}
    \item[] $X'$ debería poder particionarse en dos subconjuntos de igual suma $2\sigma$.\\
            En la subdivisión los elementos $(2\sigma - t)$ y $(\sigma + t)$ no pueden estar en la misma partición, pues su suma equivale a $3\sigma$, que excede la mitad.
            Por lo tanto, uno esta sumándose a un subconjunto $A$ y otro a un subconjunto $B$.\\
            Defino los valores: 
            \begin{itemize}
                \item $a = \sum_{ai \in A} a_i$
                \item $b = \sum_{bi \in B} b_i$
                \item sabiendo que $a + b = \sigma$
            \end{itemize}

            \begin{align}
            a + (\sigma + t) = b + (2\sigma - t)\\
            a - b = \sigma - 2t\\
            a - (\sigma - a) = \sigma - 2t\\
            2a - \sigma = \sigma - 2t\\
            2a = 2\sigma - 2t\\
            a = \sigma - t
            \end{align}

            Por lo que $b$ sería lo mismo que $\sigma - t + b = \sigma$ y esto daría como resultado $b = t$, lo que representa que $B$ es el subconjunto solución del problema del Subset-Sum. Concluyendo con esto, la existencia del Partition-Problem también garantiza la existencia del Subset-Sum
\end{enumerate}

\subsubsection{Conclusión}

Finalmente, habiendo demostrado que Partition-Problem es un problema que está en NP, y habiendo reducido un problema NP-Completo conocido (visto en clase) como Subset-Sum a Partition-Problem, se concluye que este último también es un problema NP-Completo.

\subsection{Reducción de Partition-Problem a Tribu del Agua}

Previamente, se plantearon el verificador en tiempo polinómico y el problema de decisión correspondiente a ambos casos. A continuación, se muestra la reducción: \\

\begin{center}
\textit{Partition-Problem} $\leq_{\text{p}}$ \textit{Tribu del Agua}
\end{center}

Para ello, debemos ser capaces de resolver una instancia de \textit{Partition-Problem} utilizando un algoritmo que resuelva \textit{Tribu del Agua}, empleando una cantidad de pasos polinómica.

Los parámetros que recibe \textit{Partition-Problem} son los siguientes:
\begin{itemize}
    \item $X = \{x_1,x_2, ..., x_n \}$
\end{itemize}

Estos deben transformarse de manera adecuada para que el algoritmo de \textit{Tribu del Agua} pueda resolverlo:

\begin{itemize}
    \item $\tau = \sum_{x_i \in X} x_i$
    \item $X' = X = \{x_1,x_2, ..., x_n\}$
    \item $k = 2$
    \item $B = 2(\frac{1}{2}\tau)^2 = \frac{1}{2}\tau^2$
\end{itemize}

Ahora se demuestra la ida y la vuelta de la reducción

\begin{enumerate}
    \item \underline{\textbf{Ida: Cuando hay Partition, hay Tribu}}
    \item[] Supongamos que hay Partition, entonces existen dos subconjuntos $L \subseteq X$ y $R = X - L$ tal que $\sum_{l_i \in L} l_i = \sum_{r_i \in R} r_i = \frac{1}{2}\tau$.
    
    Donde la suma de los cuadrados de ambos subconjuntos es:
    
    \begin{center}
        $(\sum_{l_i \in L} l_i)^2 + (\sum_{r_i \in R} r_i)^2 = (\frac{1}{2}\tau)^2 + (\frac{1}{2}\tau)^2 = 2(\frac{1}{2}\tau)^2 = B$
    \end{center}
    
    Por lo tanto, la instancia de Tribu del Agua se cumple.
            
    \item \underline{\textbf{Vuelta: Cuando hay Tribu, hay Partition}}
    \item[] Supongamos que hay Tribu y admite la partición en dos subgrupos $L$ y $R$ tal que

    \begin{center}
        $S_L = \sum_{l_i \in L} l_i \qquad S_R = \sum_{r_i \in R} r_i \qquad S_L^2 + S_R^2 \leq B$
    \end{center}

    y además \(S_L + S_R = \tau\). Si expandimos el cuadrado de la suma

    \begin{center}
        $(S_L + S_R)^2 = S_L^2 + S_R^2 + 2 S_L S_R = \tau^2$
    \end{center}

    De ahí se obtiene:

    \begin{center}
        $S_L^2 + S_R^2 = (S_L + S_R)^2 - 2 S_L S_R = \tau^2 - 2 S_L S_R \leq B$
    \end{center}

    Luego:
    
    \begin{center}
        $\tau^2 - 2 S_L S_R \leq \frac{1}{2}\tau^2 \longrightarrow - 2 S_L S_R \leq -\frac{1}{2}\tau^2 \longrightarrow S_L S_R \leq \frac{1}{4}\tau^2$
    \end{center}

    El producto $S_L S_R$ alcanza su máximo $\frac{1}{4}\tau^2$ únicamente cuando $S_L = S_R =  \frac{1}{2}\tau$

    Por lo tanto, la existencia de una solución para \textit{Tribu del Agua} implica la existencia de una partición de sumas iguales para \textit{Partition-Problem}
    
\end{enumerate}

\subsubsection{Conclusión}

Hemos mostrado que:

\begin{itemize}
    \item \textit{Tribu del Agua} tiene un verificador en tiempo polinómico (se demostró previamente), por lo tanto pertenece a NP.
    \item Existe una reducción en tiempo polinómico \(\textit{Partition-Problem} \le_{\text{p}} \textit{Tribu del Agua}\) cuya corrección quedó probada en ambas direcciones.
\end{itemize}

Por lo tanto, dado que \textit{Partition-Problem} es NP-Completo y existe una reducción polinómica desde él hacia \textit{Tribu del Agua}, se concluye que \textit{Tribu del Agua} también es NP-Completo.

\newpage

